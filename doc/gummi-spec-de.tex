\documentclass[12pt,openany]{book}

\usepackage[utf-8]{inputenc}
\usepackage[british,german]{babel}
\usepackage[T1]{fontenc}
\usepackage[lmargin=2cm,tmargin=2cm,rmargin=2cm,bmargin=3cm]{geometry}
\usepackage{listings}
\lstset{basicstyle=\ttfamily,columns=flexible,keywordstyle=\bfseries,
  aboveskip=\medskipamount,
  belowskip=\medskipamount}
\lstdefinelanguage{XML}{
  morekeywords={document,section,paragraph,emph},
  alsoletter={-}
}
\lstset{language=}
\usepackage{minioneu,textcomp,booktabs}
%\usepackage[T1,hyphens]{url}

\usepackage{hyperref}
  \hypersetup{%
    pdfauthor={Torsten Bronger},%
    pdftitle={Electronic properties of µc-Si:H layers investigated with Hall measurements},%
    pdfsubject={Doctoral thesis of Torsten Bronger},%
    pdfkeywords={microcrystalline silicon, Hall effect, mobility, carrier concentration}
  }
\usepackage{breakurl}

\hyphenation{Ob-jekt-Klas-sen}\sloppy
\setcounter{tocdepth}{6}
\setcounter{secnumdepth}{6}
\begin{document}
\title{Gummi Projekt-Spezifikation}
\author{Torsten Bronger}
\date{18. August 2007}
\maketitle

\tableofcontents

\selectlanguage{british}

\chapter{Gummi 1.0 specification}

\begin{quotation}
  The key words ``MUST'', ``MUST NOT'', ``REQUIRED'', ``SHALL'', ``SHALL NOT'',
  ``SHOULD'', ``SHOULD NOT'', ``RECOMMENDED'', ``MAY'', and ``OPTIONAL'' in
  this document are to be interpreted as described in
  \href{http://tools.ietf.org/html/rfc2119}{RFC~2119}.
\end{quotation}
      
\section{Gummi file format}

Gummi files are pure text files in an octet-based encoding.  This encoding
SHOULD be supported by current Python implementations, see
\url{http://docs.python.org/lib/standard-encodings.html}.  The encoding MUST
have ASCII as its \mbox{7-bit} subset.

To ensure interoperability, line endings may be those of Unix (LF), Windows
(CR+LF), or Macintosh (CR) systems.  They may even be mixed in one file.
Therefore, Gummi treats line endings like XML does: LF, CR+LF, and any CR not
followed by LF are regarded as one line ending.  (LF: ASCII~\verb|0Ah|, CR:
ASCII~\verb|0Dh|)

If not stated otherwise, whitespace characters are SPACE (ASCII:~\verb|20h|)
and TAB (ASCII:~\verb|09h|).  If not stated otherwise, a sequence of whitespace
characters is treated as one whitespace character.

\subsection{Header}

A Gummi file MAY declare Emacs-like local variables in its very first line.
This line MUST have the following format:
\begin{lstlisting}[escapechar=`]
.. -*- `\textrm{keyword$1$}`: `\textrm{value$1$}`; `\textrm{keyword$2$}`: `\textrm{value$2$}` -*-
\end{lstlisting}
All whitespace is OPTIONAL except for the very beginning of the file: It must
be ``\verb*|.. |''.

Both, keywords and values, MUST consist only of the following characters:
\verb|[A-Za-z0-9_-]|.  The values may additionally contain the comma but
\emph{not} whitespace.  Both keywords and values are treated
case-insensitively.  The following keywords are used by Gummi:

\begin{center}
\begin{tabular}{@{}lp{7cm}@{}}
  \toprule
  \verb|coding| & encoding of the file (default depends on the implementation)\\
  \verb|input-method| & input method(s) of the file, comma-separated.  Default:
                          \verb|minimal|\\
  \bottomrule
\end{tabular}
\end{center}

\noindent
Both are OPTIONAL\@.  Other keyword--value pairs MAY be declared in the line.

\bigskip
%
Immediately after the local variables in the second line of the file -- or in
the very first line if no local variables were declared -- there MAY be a Gummi
version line.  Its format is as follows:
\begin{lstlisting}[escapechar=`]
.. Gummi `$\langle$\textrm{\textit{version number}}$\rangle$
\end{lstlisting}
The version line MUST start with ``\verb*|.. |''.  The version number of this
specification is ``\verb|1.0|''.


\subsection{Comment lines}

Comment lines match the following regular expression:

\begin{verbatim}
^\.\.([ \t].*)?$
\end{verbatim}

\noindent
If not stated otherwise, this is meant by ``comment line'' throughout this
specification.

\subsection{The input method}
\label{sec:input-method}

The input method is a mechanism for entering characters that are not easily
accessible to the user, e.\,g.\ Greek letters, mathematical operators, or
special typographic characters.  The substitution rules of the input method are
applied before the parser sees the text.  However, the input method MUST NOT
remove syntactically relevant characters used for markup, including LF or~CR\@.
It may add them, though.  The replacement MUST be a single unicode character
which is not LF or~CR\@.

The default input method is called ``\verb|minimal|''.  It is used if no input
method is given in the local variables line.  The input method
``\verb|minimal|'' is specified by the reference implementation and MUST be
provided by any Gummi implementation.  The special input method name
``\verb|none|'' means that no input method is applied.

The text is searched from the beginning to the end.  If a portion of the text
matches an item of the input method, the portion is replaced with the
replacement of that item.

The following three points determine which substitution rule is applied.  If a
point means equality for two or more rules, the next point is tried.

\begin{enumerate}
\item earliest match
\item longest match
\item match declared last in the input method file(s)
\end{enumerate}

\subsection{Escaping}

Gummi uses the backslash »\char92« for escaping.  Escaping takes place on two
levels, namely the preprocessor and the parser level.  In order to avoid having
two different escaping characters, things are a little bit more complicated
than usual.

\subsubsection*{Preprocessor}

On the preprocessor level, escaping hinder input method matches from being
applied.  If a backslash is immediately before a match, the corresponding
substitution is not realised.  However, scanning for matches re-starts at the
second character of the (ignored) match.  Similarly, a backslash within a match
spoils the substitution.  However, the prepending characters may correspond to
a shorter match which is then applied.

A double backslash generates one backslash that is printed as is and that is
not used in any escaping.  It is only makred as escaped if deferred escaping
was used, see next section.  Furthermore, ``\verb|[[|'' and ``\verb|]]|'' are
transformed to \emph{escaped} ``\verb|[|'' and ``\verb|]|'', respectively

A character may be given by its decimal Unicode number of the form
\verb|\#1234;| or its hexadecimal Unicode number of the form \verb|\0x123f;|.
This has precedence over all input method matches.

\subsubsection*{Parser}

On the parser level, a backlash immediately before a character avoids using
this character for syntactic structuring.  In other words, an escaped character
is printed as is and is not used for any kind of markup.  Characters which a
used for markup in certain contexts needn't be escaped in other contexts.
Superfluous escaping is ignored.

Things are getting somewhat cumbersome if an input method match should be
escaped on the parser level.  For example, ``$\rightarrow$'' is the syntactic
symbol for cross referencing.  If you enter it as \verb|-->|, the minimal input
method replaces it with~``$\rightarrow$''.  However, if you want to have the
arrow as is, i.\,e.\ not as a syntactic command, you can't just write
\verb|\-->| because this prints as~``\mbox{-{}-{}>}''.

To overcome this problem, there is the so-called ``deferred escaping'': A
backslash before a match, with any amount of whitespace, including up to one
linebreak inbetween, escapes the character that is inserted by the
substitution.  The whitespace (not the linebreak) is removed from the input.
Thus you could write ``\verb*|\ -->|'' to get an arrow which does not begin a
cross reference.  Note that you can also simply write ``\char92$\rightarrow$''
if your editor lets you do this.

If no substitution has taken place (i.\,e., the deferred escaping was
unnecessary), the deferred escaping is just like an ordinary parser-level
escaping.  The inner whitespace is still removed, though.


\subsubsection*{Source code excerpts}

Within source code excerpts, escaping is radically different.  For escaping,
»\verb|\`|« instead of »\texttt{\char92}« is used.  However, it's only
meaningful to escape »\verb|```|« and »\verb|\`|« itself.  Apart from this, the
whole excerpt is copied as is to the output until the next~»\verb|```|«.  This
is to ensure that escaping is barely necessary in source code excerpts.

\subsection{Structuring}

~

\subsection{Cross referencing}

~

\subsection{Inline markup}

~

\subsection{Footnotes, hyperlinks, and references}

~

\subsubsection{Footnotes}

~

\subsubsection{Hyperlinks}

~

\subsubsection{References}

~

\subsection{Source code excerpts}

~

\subsection{Tables}

~

\subsection{Mathematical material}

~


\subsubsection{Physical units}

~

\subsection{Images and floating material}


~

\subsection{Directives and roles}

~

\subsubsection{Directives}

~

\subsubsection{Meta information}

~

\subsubsection{Roles}

~


\section{Input method file format}

In the first line of an input file, local variables are set analogously to the
Gummi file itself.  However, in input method files, this line is required.  It
may be for example:

\begin{lstlisting}
.. -*- input-method-name: my_method;  parental-input-method: minimal -*-
\end{lstlisting}

The following fields are known to Gummi, however, further field may be added,
they are irgnored by Gummi.

\begin{center}
\begin{tabular}{@{}lp{7cm}@{}}
  \toprule
  \verb|input-method-name| & REQUIRED; name of the input method of the file \\
  \verb|coding| & encoding of the file.  Default: \verb|utf-8|\\
  \verb|parental-input-method| & name(s) of input method(s) this method bases
  upon; comma-separated\\
  \bottomrule
\end{tabular}
\end{center}

The second line MUST be the exactly following (only additional trailing
whitespace is allowed):

\begin{lstlisting}
.. Gummi input method
\end{lstlisting}

From there on, all lines only consisting of whitespace or starting with
``\verb*|.. |'' are ignored.  All other lines are interpreted as one
substitution rule.  Every substitution rule starts with the match, followed by
one or more tabs, followed by one single character which is the replacement.
After that, the line may contain an arbitrary comment separated by whitespace.
The replacement character may also be given by its decimal Unicode number of
the form \verb|#1234| or its hexadecimal Unicode number of the form
\verb|0x123f|.

By default, the match is a simple string.  With the prefix \verb|REGEX::|,
everything after this prefix is a Python regular expression according to
\url{http://docs.python.org/lib/re-syntax.html}.  The regular expression MUST
NOT contain groups, i.\,e.\ ``\verb|(...)|'' is forbidden, while
``\verb|(?:...)|'' is allowed, for example.

By default, the substitutions are done before the parser sees the input.  In
rare cases this is unfortunate, the most prominent being ``\mbox{-{}-}'' which
is transformed to~``--''.  This breaks the parsing of tables with their
horizontal lines, as well as dashed lines which generate skips between
paragraphs.  In order to avoid this, an input method rule may be prepended by
\verb|POST::| which means that it is applied \emph{after} the parser has done
its work.  Note that this means that it cannot generate syntactic relevant
information of any kind.  Additionally, POST rules may use replacement
characters of the first pass as part of their match, unless these replecements
were escaped.

The matches (normal strings or regexps) are matched against the Gummi source
text on string level, i.\,e.\ only the proper encoding is already applied to
both the input method match and the source code.

The order of the substitution rules is significant.  The longest match wins.
If two rules match with the same length, the latest match in the Gummi file is
used.  Therefore, rules which are defined later in the input method file may
override earlier ones (or those in a parental input method), unless their match
is shorter, as explained in section~\ref{sec:input-method}.

If the local variable \verb|parental-input-method| is set, all input methods
mentioned in this variable are read in the order given in this variable.  Then,
the rules in the file are read.

The file name of an input method must be the name of the input method itself
plus the extension ``\verb|.bim|''.

\section{Additional implementation requirements}

~

\section{API for backends}

\subsection{The content models}

\begingroup\parindent0pt\parskip2ex
\newcommand{\element}[2]{\texttt{\textbf{#1}}: \texttt{#2}}

\element{Document}{\textit{block}*, Section*}

\element{Section}{Heading, \textit{block}*, Section*}

\element{Heading}{\textit{inline}}

\element{\textit{inline}}{\textit{Text} | Emphasize}

\element{Emphasize}{\textit{inline}}

\endgroup

\section{Backend requirements}

~

\selectlanguage{german}

\chapter{Referenzimplementierung}

\section{Infrastruktur}

The Projekthomepage ist \url{http://bobcat.origo.ethz.ch/}.

Die Mailingliste kann auf
\url{https://lists.sourceforge.net/lists/listinfo/latex-bronger-gummi} gefunden
werden.  Die Mailingliste ist außerdem auf Gmane gelistet unter dem
Gruppennamen \url{gmane.text.formats.gummi}.

Der Quellcode ist als SVN-Repository realisiert.  Angaben dazu finden sich auf
\url{http://bobcat.origo.ethz.ch/wiki/development}.

Ein Beispieldokument gibt es unter
\url{http://latex-bronger.sourceforge.net/gummi/gummi-example.html}.

\section{Name}

Der Name "`Gummi"' ist vorläufig.  Bessere Vorschläge sind herzlich willkommen.
Der Name sollte knuffig sein und ein tierisches Maskottchen zulassen.


\section{Projektziel}

Ziel ist, in Python einen Prototypen zu entwickeln, der die Textrepräsentierung
von Gummi einliest, in einen Abstract Syntax Tree (AST) umwandelt und an
Backends für mindestens \LaTeX\ (PDF) und HTML weitergibt.

Darüberhinaus könnte es sinnvoll weil recht einfach machbar sein, eine GUI mit
Editor und Syn\-tax-""High\-ligh\-ting drumherum zu stricken.  Meine
Erfahrungen mit wxPython haben gezeigt, daß das sehr wenig Aufwand wäre
($<500$~Zeilen).  Es soll ja keine vollständige Gummi-Entwicklungsumgebung
werden, sondern nur etwas bedienungstechnischer Zucker.


\section{Design-Philosophie des Textformats}

Das Textformat und das Umwandlungsprogramm sollen folgenden Ansprüchen genügen:

\begin{itemize}
\item Es soll intuitiv vom Benutzer einzugeben sein.
\item Es soll in der ursprünglichen Form bereits gut lesbar sein.
\item Es muß eine eindeutige Syntax haben.  Verstöße gegen diese Syntax führen
  dazu, daß das Dokument nicht (auch nicht teilweise) verarbeitet wird.  Es
  gibt eine zweite Gruppe von "`Ungereimtheiten"' in der Datei, die Warnungen
  auswirft.
\item Es ist \emph{nicht} wichtig, daß das Textformat leicht zu parsen ist.
  Hauptsache, wir Hobbyprogrammierer haben überhaupt eine Chance, diesen Parser
  zu schreiben.  Insbesondere muß die Syntax nicht systematisch sein im Sinne
  von kontextfreier Grammatik o.\,ä.  Entscheidend ist, daß sie praktisch ist.
\item Logisches Markup vor visuellem Markup, aber nicht um jeden Preis.  Klare
  Trennung von Inhalt (Gummi-Datei) und Layout (Themes), aber nicht um jeden
  Preis.
\item Zielgruppe sind alle, die nicht-triviale Dokumente mit dem Computer
  eingeben wollen.  Es geht nicht nur um die Cracks.  \LaTeX-Benutzer können
  uns erstmal ruhig verhöhnen für das, was wir so treiben.
\end{itemize}

Noch ein paar weitere Design-Eckpfeiler, inspiriert vom \emph{Zen of Python}
von Tim Peters:

\begin{itemize}
\item Mache das Häufige einfach, auf Kosten des Seltenen.

\item Würde das Seltene das Häufige nicht mehr ganz so einfach machen, erkläre
  es zum Unmöglichen.

\item Inhalt und Struktur gehören zum Dokument, aber das Layout gehört zum
  Theme.  Beides darf nicht in derselben Datei stehen.

\item Unicode ist gut, Encodings sind böse.

\item Ein Dokument wird geschrieben, nicht programmiert.  Präambeln à la LaTeX
  sind böse.

\item Syntax muß dem Autor dienen, nicht dem Compiler.

\item Es sollte immer genau einen offensichtlichen Weg geben, etwas zu tun.

\item Außer bei Tabellen.

\item Lokale Layoutanpassung ist eine schlechte Idee, globale Layoutanpassung
  kann eine gute Idee sein.

\item Kleine systembedingte Layout-Schwächen sind verzeihlich, wenn der Gewinn
  bei Benutzbarkeit und Verarbeitbarkeit der Dokumente sie rechtfertigt.

\item Erweiterungsmechanismen sind unvermeidbar, aber problematisch.  Sie
  dürfen nicht dazu führen, daß Dokumente nur unter bestimmten Umständen
  funktionieren.

\item Syntax muß viele Sprachen sprechen können.  Groß-/Kleinschreibung ist
  wichtig.

\item Gib dem Autor nicht 1001 Layout-Knöpfe zum herumspielen.  Der Autor darf
  zu seinem Layout-Glück genötigt werden.

\item Flach ist besser als verschachtelt.

\item Verarbeite nur Dokumente ohne technische Mängel.  Versuche Probleme
  automatisch zu lösen, aber nicht ohne Rückmeldung.

\item Es ist erlaubt, den Quelltext automatisch zu verändern, aber nicht ohne
  Rückmeldung.

\item Autoren lieben Themes.
\end{itemize}


\section{Programmaufbau}

Das Programm ist sowohl als Stand-Alone-Umwandlungspramm auf der Kommandozeile,
als auch als Python-Modul zu gebrauchen.   Es gliedert sich in drei Teile:

\begin{enumerate}
\item Der Parser für die Text-Repräsentation
\item Der AST, bestehend aus Klasseninstanzen
\item Die Backends, realisiert als einzelne Module
\end{enumerate}

Technisch gesehen sind die ersten beiden Punkte allerdings miteinander
verwoben, d.\,h.\ die Ob\-jekt-""Klas\-sen enthalten auch den Code, um sich selber zu
parsen.  Ich glaube allerdings nicht, daß das ein Problem ist.

\subsection{Der Abstract Syntax Tree (AST)}

Die AST ist eine Art DOM (Document Object Model), d.\,h.\ eine baumartige
Struktur aus Klassenobjekten.  Man kann sie sich wie eine XML-Datei vorstellen:

\begin{lstlisting}[language=XML]
<document>
  <section>
    <paragraph>Dies ist der <emph>erste</emph> Absatz.</paragraph>
  <section>
<document>
\end{lstlisting}

Man beachte, daß "`\texttt{Dies ist der~}"', "`\texttt{erste}"' und
"`~\texttt{Absatz.}"'\ drei sogenannte Textnodes sind, also eigene Elemente im
Baum.  Im AST von Gummi wird das genauso gehandhabt, d.\,h.\ es gibt
Text-Elemente, die keine Kinder mehr enthalten können (und insbesondere frei
von Formatierung sind; also reine Buchstabensequenzen).

Alle Klassen sind von der Klasse \lstinline{Node} abgeleitet.  Gemeinsam ist
allen, daß sie Kinder haben können (die Liste kann selbstredend leer sein), und
daß im Konstruktor die Zeilennummern des Quelltextes übergeben werden, in denen
das jeweilige Element seinen Inhalt findet, den es parsen muß und sich und
seine direkten Kinder erzeugen muß.

\bigskip
%
Die Element im AST sollten ruhig üppig ausgestattet sein, um den Backends das
Leben möglichst einfach zu machen.  So enthält die \lstinline{Section}-Klasse
sowohl die Schachtelungstiefe, als auch die Numerierung des aktuellen
Abschnitts.  Theoretisch kann das das Backend zwar selber berechnen, aber nicht
nur die Autoren von Gummi-Dokumenten, sondern auch die von Backends sollten so
weit wie möglich entlastet werden.

\subsection{Präprozessor}

Dem Parser ist der Präprozessor vorgeschaltet, der den Quelltext im richtigen
Encoding einliest und die Ersetzungen der Input Method durchführt.
Gleichzeitig führt er Buch, in welcher Zeile und in welcher Spalte dafür
Buchstaben gelöscht wurden, damit man nachher die Original-Stelle wieder
rekonstruieren kann.

Das ganze wird realisiert durch einen speziellen Datentyp \verb|Excerpt| (von
\verb|unicode| abgeleitet), welcher nicht nur den Text selber sowie den
Dateinamen und die originale Zeilennummer enthält, sondern auch Angaben dazu,
wo escapierende Backslashs standen, und wie die orginale Zeilen- und
Spaltennummer im Falle eines Fehlers wieder rekonstruiert werden kann.

\subsection{Parser}

Danach kommt dann der Parser dran.  Der Parser ist eine Menge von Klassen, die
die einzelnen Elemente des Textes repräsentieren, also z.B. \verb|Document|,
\verb|Section|, \verb|Paragraph| und \verb|Emphasize|.

Alle Querverweise, also auch solche auf Textbausteine, Fußnoten und annotierte
Hyperlinks, werden erstmal als Referenzen eingefügt.  Nachdem das Dokument
vollständig prozessiert wurde, werden Textbausteine, Fußnoten und annotierte
Hyperlinks rekursiv aufgelöst, wobei zyklische Inklusionen natürlich einen
Fehler verursachen.  „Aufgelöst“ bedeuet in diesem Fall, daß sie an die
referenzierenden Stellen im AST hineinkopiert werden.

Literaturstellen und Meta-Daten werden an einer wurzelnaher Stelle des AST
gesammelt.

\subsection{Backends}

Ein Backend ist für genau ein Ausgabeformat zuständig.  Das Backend wird
irgendwann aufgefordert, seine Callback-Funktionen in den AST zu injizieren.
Das bedeutet, daß es Methoden in allen Klassen setzt bzw.\ von einem
Vorgänger-Run u.\,U. eines anderen Backends überschreibt.  Gummi ruft dann die
Callback-Methode der Top-Node auf, die dann verschachtelt den ganzen Baum
abarbeitet und so die Ausgabe erzeugt.

Das Backend kann dann auch noch weitere Arbeiten erfüllen, z.\,B. pdfLaTeX,
xindy etc.\ aufrufen.


\section{Themes}

Benutzer lieben Themes.  Bei Gummi sind Themes Gruppen von Backends.  Ein Theme
"`Doktorarbeit WZL"' kann also aus einem \LaTeX/PDF und HTML-Backend bestehen.
Ist ein Ausgabeformat in einem Theme nicht vorhanden, gibt's ein Fallback mit
einer Warnung bzw.\ einen Fehler, wenn ein Ausgabeformat von keinem
registrierten Backend abgedeckt wird.

Für bestimmte Ausgabeformate wird es Helfer-Module geben, die Code enthalten,
der für alle Themes, die dieses Ausgabeformat unterstützen, nützlich sein
können.  Außerdem kann ein Backend ein anderes Backend sich erstmal
installieren lassen und dann noch ein paar eigene Dinge drüberbügeln.  Gerade
für \LaTeX\ werden die Backends ja zu 90\,\% gleiche Dinge tun, im Zweifel wird
nur eine andere Präambel benutzt.

Das Standard-Theme heißt "`Standard"'.

Man kann dem Theme -- bzw.\ letztlich dem Backend -- Optionen übergeben, genau
wie im optionalen Parameter von \verb|\includegraphics|.  Damit kann man
Papiergröße, Schriftart etc.\ einstellen.  Es wird Empfehlungen für die
wichtigen Optionsnamen geben, damit nicht ein Backend die Papiergröße mit
"`paper~size"' und ein anderes mit "`papersize"' einstellen läßt.  Auch hier
gilt wieder: Englisch geht immer, außerdem eventuell die aktuell eingestellte
Dokumentsprache.

\section{Writebacks}

Etwas, was in \LaTeX\ verpönt wäre, ist in Gummi möglich, nämlich daß der
Umwandler bzw.\ der Editor den Quellcode verändert.  Zur Zeit passiert das an
zwei Stellen, aber nur, wenn der Benutzer dies ausdrücklich wünscht:

\begin{enumerate}
\item Die Numerierung der Abschnitte, Gleitumgebungen, Listen etc.\ wird
  angepaßt.
\item Die Unicode-Ersetzungen aus der Input-Method werden in den Quellcode
  hineingeschrieben.
\end{enumerate}

Man könnte sich auch vorstellen, daß beim Writeback Fußnoten, Linklisten und
Literaturstellen an bessere Stellen im Quelltext verschoben werden.

\end{document}

% LocalWords:  Cracks Themes
