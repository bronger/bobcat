\documentclass[12pt,openany]{book}

\usepackage[utf8]{inputenc}
\usepackage[british]{babel}
\usepackage[T1]{fontenc}
\usepackage[lmargin=2cm,tmargin=2cm,rmargin=2cm,bmargin=3cm]{geometry}
\usepackage{listings}
\lstset{basicstyle=\ttfamily,columns=flexible,keywordstyle=\bfseries,
  aboveskip=\medskipamount,
  belowskip=\medskipamount}
\lstdefinelanguage{XML}{
  morekeywords={document,section,para,emph},
  alsoletter={-}
}
\lstset{language=}
\usepackage{minioneu,textcomp,booktabs,mdwlist}
%\usepackage[T1,hyphens]{url}

\usepackage{hyperref}
  \hypersetup{%
    pdfauthor={Torsten Bronger},%
    pdftitle={Bobcat secification}%
  }

\setcounter{tocdepth}{6}
\setcounter{secnumdepth}{6}
\sloppy
\begin{document}
\title{Bobcat project specification}
\author{Torsten Bronger}
\date{18. August 2007}
\maketitle

\tableofcontents

\chapter{Bobcat 1.0 specification}

\begin{quotation}
  The key words ``MUST'', ``MUST NOT'', ``REQUIRED'', ``SHALL'', ``SHALL NOT'',
  ``SHOULD'', ``SHOULD NOT'', ``RECOMMENDED'', ``MAY'', and ``OPTIONAL'' in
  this document are to be interpreted as described in
  \href{http://tools.ietf.org/html/rfc2119}{RFC~2119}.
\end{quotation}
      
\section{Bobcat file format}

Bobcat files are pure text files in an octet-based encoding.  This encoding
SHOULD be supported by current Python implementations, see
\url{http://docs.python.org/lib/standard-encodings.html}.  The encoding MUST
have ASCII as its \mbox{7-bit} subset.  Bobcat files MUST NOT contain NULL
characters.

To ensure interoperability, line endings may be those of Unix (LF), Windows
(CR+LF), or Macintosh (CR) systems.  They may even be mixed in one file.
Therefore, Bobcat treats line endings like XML does: LF, CR+LF, and any CR not
followed by LF are regarded as one line ending.  (LF: ASCII~\verb|0Ah|, CR:
ASCII~\verb|0Dh|)

If not stated otherwise, whitespace characters are SPACE (ASCII:~\verb|20h|)
and TAB (ASCII:~\verb|09h|).  If not stated otherwise, a sequence of whitespace
characters is treated as one whitespace character.

\subsection{Header}

A Bobcat file MAY declare Emacs-like local variables in its very first line.
This line MUST have the following format:
\begin{lstlisting}[escapechar=`]
.. -*- `\textrm{keyword$1$}`: `\textrm{value$1$}`; `\textrm{keyword$2$}`: `\textrm{value$2$}` -*-
\end{lstlisting}
All whitespace is OPTIONAL except for the very beginning of the file: It must
be ``\verb*|.. |''.

Both, keywords and values, MUST consist only of the following characters:
\verb|[A-Za-z0-9_-]|.  The values may additionally contain the comma but
\emph{not} whitespace.  Both keywords and values are treated
case-insensitively.  The following keywords are used by Bobcat:

\begin{center}
\begin{tabular}{@{}lp{7cm}@{}}
  \toprule
  \verb|coding| & encoding of the file (default depends on the implementation)\\
  \verb|input-method| & input method(s) of the file, comma-separated.  Default:
                          \verb|minimal|\\
  \bottomrule
\end{tabular}
\end{center}

\noindent
Both are OPTIONAL\@.  Other keyword--value pairs MAY be declared in the line.

\bigskip
%
Immediately after the local variables in the second line of the file -- or in
the very first line if no local variables were declared -- there MAY be a Bobcat
version line.  Its format is as follows:
\begin{lstlisting}[escapechar=`]
.. Bobcat `$\langle$\textrm{\textit{version number}}$\rangle$
\end{lstlisting}
The version line MUST start with ``\verb*|.. |''.  The version number of this
specification is ``\verb|1.0|''.


\subsection{Comment lines}

Comment lines match the following regular expression:

\begin{verbatim}
^\.\.([ \t].*)?$
\end{verbatim}

\noindent
If not stated otherwise, this is meant by ``comment line'' throughout this
specification.

\subsection{The input method}
\label{sec:input-method}

The input method is a mechanism for entering characters that are not easily
accessible to the user, e.\,g.\ Greek letters, mathematical operators, or
special typographic characters.  The substitution rules of the input method are
applied before the parser sees the text.  However, the input method MUST NOT
remove syntactically relevant characters used for markup, including LF or~CR\@.
It may add them, though.  The replacement MUST be a single unicode character
which is not LF or~CR\@.

The default input method is called ``\verb|minimal|''.  It is used if no input
method is given in the local variables line.  The input method
``\verb|minimal|'' is specified by the reference implementation and MUST be
provided by any Bobcat implementation.  The special input method name
``\verb|none|'' means that no input method is applied.

The text is searched from the beginning to the end.  If a portion of the text
matches an item of the input method, the portion is replaced with the
replacement of that item.

The following three points determine which substitution rule is applied.  If a
point means equality for two or more rules, the next point is tried.

\begin{enumerate}
\item earliest match
\item longest match
\item match declared last in the input method file(s)
\end{enumerate}

\subsection{Escaping}

Bobcat uses the backslash »\char92« for escaping.  Escaping takes place on two
levels, namely the preprocessor and the parser level.  In order to avoid having
two different escaping characters, things are a little bit more complicated
than usual.

\subsubsection*{Preprocessor}

On the preprocessor level, escaping hinder input method matches from being
applied.  If a backslash is immediately before a match, the corresponding
substitution is not realised.  However, scanning for matches re-starts at the
second character of the (ignored) match.  Similarly, a backslash within a match
spoils the substitution.  However, the prepending characters may correspond to
a shorter match which is then applied.

A double backslash generates one backslash that is printed as is and that is
not used in any escaping.  It is only marked as escaped if deferred escaping
was used, see next section.  Furthermore, ``\verb|[[|'' and ``\verb|]]|'' are
transformed to \emph{escaped} ``\verb|[|'' and ``\verb|]|'', respectively

A character may be given by its decimal Unicode number of the form
\verb|\#1234;| or its hexadecimal Unicode number of the form \verb|\0x123f;|.
This has precedence over all input method matches.

\subsubsection*{Parser}

On the parser level, a backlash immediately before a character avoids using
this character for syntactic structuring.  In other words, an escaped character
is printed as is and is not used for any kind of markup.  Characters which a
used for markup in certain contexts needn't be escaped in other contexts.
Superfluous escaping is ignored.

Things are getting somewhat cumbersome if an input method match should be
escaped on the parser level.  For example, ``$\rightarrow$'' is the syntactic
symbol for cross referencing.  If you enter it as \verb|-->|, the minimal input
method replaces it with~``$\rightarrow$''.  However, if you want to have the
arrow as is, i.\,e.\ not as a syntactic command, you can't just write
\verb|\-->| because this prints as~``\mbox{-{}-{}>}''.

To overcome this problem, there is the so-called ``deferred escaping'': A
backslash before a match, with any amount of whitespace, including up to one
linebreak inbetween, escapes the character that is inserted by the
substitution.  The whitespace (not the linebreak) is removed from the input.
Thus you could write ``\verb*|\ -->|'' to get an arrow which does not begin a
cross reference.  Note that you can also simply write ``\char92$\rightarrow$''
if your editor lets you do this.

If no substitution has taken place (i.\,e., the deferred escaping was
unnecessary), the deferred escaping is just like an ordinary parser-level
escaping.  The inner whitespace is still removed, though.


\subsubsection*{Source code excerpts}

Within source code excerpts, escaping is radically different.  For escaping,
»\verb|\`|« instead of »\texttt{\char92}« is used.  However, it's only
meaningful to escape »\verb|```|« and »\verb|\`|« itself.  Apart from this, the
whole excerpt is copied as is to the output until the next~»\verb|```|«.  This
is to ensure that escaping is barely necessary in source code excerpts.

\subsection{Structuring}

~

\subsection{Cross referencing}

~

\subsection{Inline markup}

~

\subsection{Footnotes, hyperlinks, and references}

~

\subsubsection{Footnotes}

~

\subsubsection{Hyperlinks}

~

\subsubsection{References}

~

\subsection{Source code excerpts}

~

\subsection{Tables}

~

\subsection{Mathematical material}

~


\subsubsection{Physical units}

~

\subsection{Images and floating material}


~

\subsection{Directives and roles}

~

\subsubsection{Directives}

~

\subsubsection{Meta information}

~

\subsubsection{Roles}

~


\section{Input method file format}

In the first line of an input file, local variables are set analogously to the
Bobcat file itself.  However, in input method files, this line is required.  It
may be for example:

\begin{lstlisting}
.. -*- input-method-name: my_method;  parental-input-method: minimal -*-
\end{lstlisting}

The following fields are known to Bobcat, however, further field may be added,
they are ignored by Bobcat.

\begin{center}
\begin{tabular}{@{}lp{7cm}@{}}
  \toprule
  \verb|input-method-name| & REQUIRED; name of the input method of the file \\
  \verb|coding| & encoding of the file.  Default: \verb|utf-8|\\
  \verb|parental-input-method| & name(s) of input method(s) this method bases
  upon; comma-separated\\
  \bottomrule
\end{tabular}
\end{center}

The second line MUST be the exactly following (only additional trailing
whitespace is allowed):

\begin{lstlisting}
.. Bobcat input method
\end{lstlisting}

From there on, all lines only consisting of whitespace or starting with
``\verb*|.. |'' are ignored.  All other lines are interpreted as one
substitution rule.  Every substitution rule starts with the match, followed by
one or more tabs, followed by one single character which is the replacement.
After that, the line may contain an arbitrary comment separated by whitespace.
The replacement character may also be given by its decimal Unicode number of
the form \verb|#1234| or its hexadecimal Unicode number of the form
\verb|0x123f|.

By default, the match is a simple string.  With the prefix \verb|REGEX::|,
everything after this prefix is a Python regular expression according to
\url{http://docs.python.org/lib/re-syntax.html}.  The regular expression MUST
NOT contain groups, i.\,e.\ ``\verb|(...)|'' is forbidden, while
``\verb|(?:...)|'' is allowed, for example.

By default, the substitutions are done before the parser sees the input.  In
rare cases this is unfortunate, the most prominent being ``\mbox{-{}-}'' which
is transformed to~``--''.  This breaks the parsing of tables with their
horizontal lines, as well as dashed lines which generate skips between
paragraphs.  In order to avoid this, an input method rule may be prepended by
\verb|POST::| which means that it is applied \emph{after} the parser has done
its work.  Note that this means that it cannot generate syntactic relevant
information of any kind.  Additionally, POST rules may use replacement
characters of the first pass as part of their match, unless these replacements
were escaped.

The matches (normal strings or regexps) are matched against the Bobcat source
text on string level, i.\,e.\ only the proper encoding is already applied to
both the input method match and the source code.

The order of the substitution rules is significant.  The longest match wins.
If two rules match with the same length, the latest match in the Bobcat file is
used.  Therefore, rules which are defined later in the input method file may
override earlier ones (or those in a parental input method), unless their match
is shorter, as explained in section~\ref{sec:input-method}.

If the local variable \verb|parental-input-method| is set, all input methods
mentioned in this variable are read in the order given in this variable.  Then,
the rules in the file are read.

The file name of an input method must be the name of the input method itself
plus the extension ``\verb|.bim|''.

\section{Additional implementation requirements}

~

\section{API for backends}

\subsection{The content models}

\makeatletter
\newcommand*{\contentmodel}[3]{\bigskip\bigskip\begingroup\par\noindent
  \fbox{\vbox{\raggedright\parindent0pt\hangafter1\hangindent1em
      \emph{Content model of} \texttt{\textbf{#1}}:\quad
      \texttt{#2}%
      \ifx#3\@empty\else\par\hangafter1\hangindent1em
      \emph{Further constraints}:\quad#3\par\fi}}\par
  \endgroup\vspace*{1ex}\par\noindent\ignorespaces}
\newenvironment{attributes}[1][Attributes]{\noindent\emph{#1:}\vspace*{-1ex}\begin{description*}}{\end{description*}}
\newcommand{\attribute}[3]{\item[\texttt{#1} \textmd{(#2)}]\hskip0.5em #3}
\newenvironment{key-values}{\vspace*{-0.8ex}\begin{description*}}{\end{description*}}
\newcommand{\keyvalue}[3]{\item[\texttt{#1} \textmd{(#2)}]\hskip0.5em #3}
\makeatother
\newcommand{\rep}{\raisebox{-0.3ex}{\kern0.1em*}}
\newcommand{\opt}{\kern0.1em?}
\newcommand{\repp}{\kern0.1em+}

\begin{attributes}[Common attributes]
  \attribute{language}{unicode}{human language in RFC\,4646 form}
\end{attributes}


\contentmodel{Document}{\textit{flow}\rep, (Section | Part)\rep, Appendix\opt}{}
\begin{attributes}
  \attribute{features}{set of unicode}{features needed by this document,
    e.\,g.\ tables, formulae etc.  This can be used by the \LaTeX\ backend to
    decide which additional packages must be included.  By convention, it
    simply contains \LaTeX\ package names to denote what's being needed, but
    maybe RTF or OpenDocument need further possible items in this set.

    An important application of it is that its}
\end{attributes}

\contentmodel{\textit{block}}{Math | Environment | Table | Code | List |
  DefinitionList | Paragraph | Footnote | Media}{}

~

\contentmodel{\textit{flow}}{\textit{block} | Float | BibliographyEntry}{}

~

\contentmodel{\textit{inline}}{\textit{Text} | Emphasize | Hyperlink | Footnote |
  IndexEntry | CrossReference | Citation | Math | Role | Media | IndexEntry}{}

~

\contentmodel{DefinitionList}{DefinitionListItem\rep}{}

~

\contentmodel{DefinitionListItem}{\textit{Text}, \textit{block}\rep}{}

~

\contentmodel{List}{ListItem\rep}{}

~

\contentmodel{ListItem}{\textit{block}\rep}{}

~

\contentmodel{Paragraph}{\textit{inline}\rep}{}

~

\contentmodel{Section}{Heading, \textit{flow}\rep, Section\rep}{}
\begin{attributes}
  \attribute{nesting\_level}{int}{The nesting depth of the section within the
    sectioning hierarchy.  The top-level sections have a value of~$0$,
    subsections have~$1$ etc.  Parts have a value of~$-1$.  Thus, it is very
    similar to the secnumdepth value in \LaTeX's book document class.

    However, the top-level sections (i.\,e., those with exactly one dot in
    their section numbering) \emph{always} have a nesting level of~$0$.  They
    may become chapters or sections in the \LaTeX\ output, depending on the
    theme and the options passed to it.}
  \attribute{suppress\_numbering}{bool}{if \lstinline{True}, don't generate a
    number for this section.}
\end{attributes}

\contentmodel{Heading}{\textit{inline}\rep}{}

~

\contentmodel{Part}{Heading}{}

~

\contentmodel{Appendix}{Section\rep}{}

~

\contentmodel{Footnote}{\textit{inline}\rep}{footnotes must not be nested}

~

\contentmodel{Emphasize}{\textit{inline}\rep}{}

~

\contentmodel{Hyperlink}{\textit{inline}\rep}{hyperlinks must not be nested}
\begin{attributes}
  \attribute{url}{unicode}{the target URL of the hyperlink}
\end{attributes}
  

\contentmodel{Role}{\textit{inline}\rep}{roles must not be nested}
\begin{attributes}
  \attribute{name}{unicode}{the name of the role}
\end{attributes}


\contentmodel{Citations}{Citation\rep}{}

~

\contentmodel{Citation}{BibliographyEntry, \textit{inline}\rep}{citations must not be nested}

~

\contentmodel{BiblioEntry}{\textit{empty}}{}

~

\contentmodel{CrossReference}{\textit{inline}\rep}{cross references must not be nested}

~

\contentmodel{Code}{\textit{Text}}{}

~

\contentmodel{Environment}{EnvironmentHeading\opt, \textit{block}\rep}{environments must not be nested}

~

\contentmodel{EnvironmentHeading}{\textit{inline}\rep}{}

~

\contentmodel{Float}{Caption\opt, (Subfigure\rep\ | \textit{block}\rep)}{}

~

\contentmodel{Subfigure}{Caption\opt, \textit{block}\rep}{}

~

\contentmodel{Caption}{\textit{inline}\rep}{}

~

\contentmodel{Media}{\textit{empty}}{}
\begin{attributes}
  \attribute{url}{unicode}{the source URL where this media object can be found}
  \attribute{options}{dict}{options given by the user.  Possible items:
    \begin{key-values}
      \keyvalue{scaling}{float}{the scaling percentage}
    \end{key-values}
  }
\end{attributes}


\contentmodel{Math}{<FixMe: To be done>}{}
\begin{attributes}
  \attribute{block}{bool}{\lstinline{True} if it's a block equation,
    \lstinline{False} if it's an inline equation.}
\end{attributes}


\contentmodel{IndexEntry}{IndexEntryPart\kern0.1em\{1,3\}}{}

~

\contentmodel{IndexEntryPart}{\textit{inline}\rep}{}

~

\section{Backend requirements}

~


\chapter{Reference implementation}

\section{Infrastructure}

The project's home page is at \url{http://bobcat.origo.ethz.ch/}.

The mailing list is on
\url{https://lists.sourceforge.net/lists/listinfo/latex-bronger-gummi}.  It is
also listed on Gmane with the group name \url{gmane.text.formats.gummi}.

The SVN repository of the source code is at
\url{http://bobcat.origo.ethz.ch/wiki/development}.  You can also browse through
the interfaces at \url{http://latex-bronger.sourceforge.net/bobcat/}.


\section{Goals}

The primary goal is to generate a prototype in Python that reads a Bobcat
source code, converts it internally into an Abstract Source Tree (AST), and
pass this to backends for at least PDF (via \LaTeX) and HTML\@.

Moreover, it may be feasible and sensible to develop a GUI with editor and
syntax highlighting.  My experiences with wxPython showed that this can be done
with less than 1000 lines of code, at least for the basic stuff.


\section{Design guidelines of the Bobcat text format}

The text format is designed along the following guidelines:

\begin{itemize}
\item It is supposted to be entered intuitively by the author.
\item It is supposed to be well-legible in its original form.
\item It must have an unambiguous syntax.  Syntax errors are fatal, however,
  there is a second class of inconsistencies which generates warnings only.
\item It is \emph{not} important that the format can be parsed easily.  Well,
  it must be realistic that amateur programmers can write a parser for it, but
  that's all.  The important thing is that the syntax is practical for most
  authors.
\item Semantic markup takes precedence over visual markup but not at all costs.
  Clear separation between contents (Bobcat file) and layout (themes) but not at
  all costs.
\item The target group are all authors who want to write non-trivial documents
  with the computer.  It's not for the geeks.
\end{itemize}

Yet another set of design guidelines, inspired from Tim Peter's \emph{Zen of Python}:

\begin{itemize}
\item Make the frequent things simple at the expense of the rare things.

\item If the rare thing made the frequent thing less simple, declare it
  impossible.

\item Contents and structure belong to the document but the layout belongs to
  the theme.  Both must not be kept within the same file.

\item Unicode is good, encoding are evil.

\item A document is written and not programmed.  Preambles à la \LaTeX\ are
  evil.

\item Syntax must serve the author, not the parser.

\item There should be one obvious way to do it.

\item Except for tables.

\item Local layout adjustments are a bad idea.  Global layout adjustments may
  be a good idea.

\item Minor systematic weaknesses in the layout are acceptable if justified by
  the gain in usability and processability of the documents.

\item Extension mechanisms are unavoidable but problematic.  They must not make
  some documents work under certain conditions only.
  
\item Syntax must be multi-lingual.  Case sensitivity is important.
  
\item Don't give the author 1001 layout knobs to play with.  Good layout may be
  gently forced.
  
\item Flat is better than nested.

\item Process only documents without syntax errors.  Try to solve problems
  automatically but not without asking the author.

\item It is allowed to change documents with the autor's consent.

\item People love themes.
\end{itemize}


\section{Structure of the program code}

The program should be useable as both a stand-alone command line tool and a
Python module.  It consists of three parts:

\begin{enumerate}
\item The parser,
\item the AST, consisting of class instances,
\item the backends, realised as Python modules.
\end{enumerate}

Technically, the first two points are intertwined, i.e., the classes also
contain the code to parse them and their children.

\subsection{The abstract syntax tree (AST)}

The AST is some kind of DOM (document object model), i.e. a tree-like structure
made of class instances.  You may see them as somethin like an XML file:

\begin{lstlisting}[language=XML]
<document>
  <section>
    <para>This is the <emph>first</emph> paragraph.</para>
  <section>
<document>
\end{lstlisting}

Note that ``\texttt{This is the~}'', ``\texttt{first}'', and
``~\texttt{paragraph.}''\ are three so-called text nodes in XML, which are
separate elements in the tree.  This is the same in Bobcat: There are text
nodes, which can't contain any children any more.  They are pure character
sequences.

All classes are derived from the class \lstinline{Node}.  They have in common
to be able to have children (this list may be empty of course), and some
methods for parsing and backend-processing.

\bigskip
%
The elements in the AST should have a large set of attributes and special
method so that the backend gets a rich and convenient environment.  For
example, the \lstinline{Section} class can give its nesting level although the
backend could find it out itself by parsing the section number.  However, not
only Bobcat authors but also backend authors should live as comfortable as
possible.

\subsection{Preprocessor}

There is a preprocessor before any parsing.  Its purpose is twofold:

\begin{enumerate}
\item Read in the source code and transform it to a Unicode-string-like data
  structure (called \lstinline{Excerpt}) that also records escaped characters,
  and the original file positions of every character (for generating verbose
  error messages).
\item Perform the substitutions of the input method.
\end{enumerate}


\subsection{Parser}

All cross references, including those to text blocks, footnotes and annoted
hyperlinks are included as references at first.  After the document has been
processed fully, these references are resolved, i.e.\ they are copied to the
referencing elements in the AST\@.

Bibliographic references and meta data is collected in the root node of the
AST, an instance of the class \lstinline{Document}.


\subsection{Backends}

A backend is responsible for exactly one output file format.  The backend's
routines are injected into the classes of the AST, so they override dummy
methods in the AST classes.  Then, the process method of the root node is
called which then calls recursively all process methods of the tree.  By this,
the final output (e.g.\ the \LaTeX\ file) is generated.

The backend may contain additional code, e.g.\ for generating PDF from the
\LaTeX\ file, or for calling Bib\TeX\ or xindy, or for cleaning up the working
directory.


\section{Themes}

Users love themes.  In Bobcat, themes are groups of backends.  Thus, a theme
``PhD thesis IEF-5'' may consist of a \LaTeX/PDF and an HTML backend.  Is a
certain output format not available in a theme, there's a fallback with a
warning.

For some backends, there will be auxiliary modules that contain code which
is useful for all themes that implement the respective backend.  Moreover, a
backend can install another backend first and then just override some things.
For example, the \LaTeX\ backends will differ only very slightly, maybe even in
the preamble.

The standard theme is called ``Standard''.

You may pass options to the theme -- effectively to the backend of that theme
-- in the Bobcat source document.  So some things can be set to personal
preferences, like paper size, font etc.  There will be a core set of options
with specified semantics, however, each backend may support additional
options.  English option names work always, the current document language may
also work.


\section{Writebacks}

In the Bobcat world, the converter or the Bobcat editor is allowed to change the
source text but only if the user doesn't object to it.  There are two purposes
for this: First, the section, enumeration etc numbers may be adjusted in order
to have the same numbers in the document and in the PDF, and the input method
substitutions may be written back to the source file.

Additionally, in case of a missing encoding information, the Bobcat converter
will perform an autodetect and may write the found encoding into the source.

\end{document}

% LocalWords:  Cracks Themes Bobcat regexps Emphasize AST backend's IEF
